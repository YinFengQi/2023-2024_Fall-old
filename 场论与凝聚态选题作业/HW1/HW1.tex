\documentclass{ctexart}
\usepackage{graphicx}
\usepackage{geometry}
\usepackage{amsfonts,amssymb,amsthm,amsmath,physics,extarrows}
\usepackage{mathrsfs}
\usepackage{tikz}
\usepackage{hyperref}% 这样可以有超链接(可以点击翻页)
\usepackage{simpler-wick}% wick收缩
% \usepackage{arydshln}% 虚线

\usetikzlibrary{arrows.meta,decorations.markings,calc,bending}



\newtheorem{definition}{定义}[section]
\newtheorem{lemma}{引理}[section]
\newtheorem{theorem}{定理}[section]
\newtheorem{proposition}{命题}[section]
\newtheorem{example}{例}[section]

\newcommand{\bm}[1]{\boldsymbol{#1}}
%这行是用来支持latex-workshop预览的

\numberwithin{equation}{section}

% 来自Castel的图片插入代码
\usepackage{import}
\usepackage{pdfpages}
\usepackage{transparent}
\usepackage{xcolor}

\newcommand{\incfig}[2][1]{%
    \def\svgwidth{#1\columnwidth}
    \import{./figures/}{#2.pdf_tex}
}

% \pdfsuppresswarningpagegroup=1 这一行似乎只在pdftex里使用

	% 请检查相对路径是否有效
\title{第一次作业}
\author{mny}

\begin{document}

\maketitle

\section{Path Integral without $\int \mathrm{d} p$}
标准的路径积分
\begin{equation}
  \begin{gathered}
    \bra{x_N}U(t_N, t_0)\ket{x_0} = \left( \prod_{n=1}^{N-1} \int \frac{\mathrm{d} p_{n+ \frac{1}{2}} \, \mathrm{d} x_n}{2\pi \hbar} \right) \int  \frac{\mathrm{d} p_{\frac{1}{2}}}{2\pi\hbar} 
    \\
    \times 
    \exp\left( 
        \frac{\mathrm{i} }{\hbar} \sum_{n=0}^{N-1} \left[ p_{n+ \frac{1}{2}}(x_{n+1}- x_n) - \delta t \frac{H(p_{n+\frac{1}{2}},x_{n+1}, t_{n+ \frac{1}{2}}) + H(\cdots, x_{n},\cdots)}{2} \right] 
     \right) ,
  \end{gathered}
\end{equation}
其中动量的部分是高斯型的, 我们把它写出来
\begin{equation}\label{integral on momenta}
  \prod_{n=0}^{N-1} \left[ 
    \int \frac{\mathrm{d} p_{n+ \frac{1}{2}}}{2\pi\hbar} \exp \left( 
        \frac{\mathrm{i} }{\hbar}\left[ p_{n+ \frac{1}{2}}\delta x_n - \delta t \left( \frac{p^2}{2m}+ \frac{V(x_{n+1}+V(x_n))}{2} \right)  \right] 
     \right) 
   \right] .
\end{equation}
应用高斯积分公式
\begin{equation}
  \int_{-\infty}^{\infty} \mathrm{e}^{-a x^2 + bx} \, \mathrm{d}x = \sqrt{\frac{\pi}{a}} \mathrm{e}^{ \frac{b^2}{4a}}.
\end{equation}
对应到这里,
\begin{equation}
  \begin{cases} 
    a = \frac{\mathrm{i} \delta t}{\hbar} \frac{1}{2m}
    \\ 
    b = \frac{\mathrm{i} \delta x_n}{\hbar}
  \end{cases}
\end{equation}
于是\eqref{integral on momenta}的结果为
\begin{equation}
  \prod_{n=0}^{N-1} \frac{1}{2\pi\hbar} \sqrt{\frac{2\pi \hbar m}{\mathrm{i} \delta t}} \mathrm{e}^{\frac{\mathrm{i} (\delta x_n)^2 m}{2\hbar \delta t}} 
  \mathrm{e}^{- \frac{\mathrm{i} }{\hbar}\delta t \frac{V(x_{n+1}+ V(x_n))}{2}}
\end{equation}

路径积分去掉动量之后的结果为
\begin{equation}
  \bra{x_{N}}U(t_N, t_0) \ket{x_0} = \left( \prod_{n=1}^{N-1} \int \mathrm{d} x_n \right) 
  \left( \prod_{n=0}^{N-1} \sqrt{\frac{m}{2\pi \mathrm{i} \hbar \delta t }}\mathrm{e}^{ \frac{\mathrm{i} \delta t}{\hbar}\left[ \frac{m}{2} \left( \frac{\delta x_n}{\delta t} \right) ^2 - \frac{V(x_{n+1}) + V(x_n)}{2} \right] }  \right) 
\end{equation}
也可以写成紧致的形式
\begin{equation}
  \bra{x_N}U(t_N, t_0)\ket{x_0} = \sqrt{\frac{m}{2\pi \mathrm{i}  \hbar \delta t}} \prod_{n=1}^{N-1} \left( \int\mathrm{d} x_n \, \sqrt{\frac{m}{2\pi \mathrm{i}  \hbar \delta t}} \right) \mathrm{e}^{\frac{\mathrm{i} }{\hbar}\int \mathrm{d} t \,L}
\end{equation}

相对论性的哈密顿量$H = \sqrt{p^2 + m^2}$. 对指数上的项$p \dot{x} - \sqrt{p^2 + m^2}$使用鞍点近似(saddle approximation),
带入
\begin{equation}
  p = \frac{m \dot{x}}{\sqrt{1 - \dot{x}^2}} + p_{(1)},
\end{equation}
这一项变为
\begin{equation}
  \begin{aligned}
    p \dot{x} - H  
    & = \frac{m \dot{x}^2}{\sqrt{1- \dot{x}^2}}+ p_{(1)} \dot{x}
    - \sqrt{m^2 + \frac{m^2 \dot{x}^2}{1- \dot{x} ^2}+ 2 \frac{m \dot{x}}{\sqrt{1- \dot{x}^2}}p_{(1)} + p_{(1)}^2}
    \\
    & = \frac{m \dot{x}^2}{\sqrt{1- \dot{x}^2}}+ p_{(1)} \dot{x}
    - \frac{m}{\sqrt{1- \dot{x}^2}}\sqrt{ 1 + \frac{2 \dot{x} \sqrt{1- \dot{x}^2}}{m} p_{(1)} + \frac{1- \dot{x}^2}{m^2} p_{(1)}^2}
    \\
    & \kern -5em   = \frac{m \dot{x}^2}{\sqrt{1-\dot{x}^2}}- \frac{m}{\sqrt{1- \dot{x}^2}} + p_{(1)}\dot{x} - p_{(1)} \dot{x} 
    -\frac{m \dot{x}^2}{\sqrt{1- \dot{x}^2}} \frac{1-\dot{x}^2}{2m^2}p_{(1)}^2 - \frac{m}{\sqrt{1- \dot{x}^2}}\left( -\frac{1}{8} \right) \frac{4 \dot{x}^2 (1-\dot{x}^2)}{m^2}p_{(1)}^2
    \\
    & = -m \sqrt{1-\dot{x}^2}+ \left( 1- \dot{x}^2 \right) ^{\frac{3}{2}} \frac{p_{(1)}^2}{2m}
  \end{aligned}
\end{equation}
于是, 含有动量的积分为
\begin{equation}
  \prod_{n=0}^{N-1} \left[ \int\frac{\mathrm{d} p_{n+ \frac{1}{2}}}{2\pi\hbar} \exp\left( \frac{\mathrm{i} \delta t}{\hbar}  \left[ -m \sqrt{1-\dot{x}^2} + \left( 1- \dot{x}^2 \right) ^{\frac{3}{2}} \frac{p_{(1)}^2}{2m} \right]  \right)  \right] 
\end{equation}
完成积分, 它变为
\begin{equation}
  \prod_{n=0}^{N-1} \left(  \sqrt{\frac{ m }{2\pi\mathrm{i} \hbar  \delta t (1-\dot{x}^2)^{\frac{3}{2}}}} \mathrm{e}^{- \frac{\mathrm{i} \delta t }{\hbar}m \sqrt{1- \dot{x}^2}} \right) 
\end{equation}
最终的结果是
\begin{equation}
  \bra{x_N}U(t_N, t_0)\ket{x_0} = \sqrt{\frac{m}{2\pi \mathrm{i} \hbar \delta t}} \frac{1}{\left( 1-\dot{x}_N^2 \right) ^{\frac{4}{3}}} \prod_{n=1}^{N-1} 
  \left( 
    \sqrt{\frac{m}{2\pi \mathrm{i} \hbar \delta t}} \frac{\mathrm{d} x_n}{\left( 1- \dot{x}_n^2 \right) ^{\frac{4}{3}}} \mathrm{e}^{- \frac{\mathrm{i} \delta t}{\hbar}m \sqrt{1-\dot{x}^2}}
   \right) ,
\end{equation}
其中$\dot{x}_n \equiv \frac{x_n - x_{n-1}}{\delta t}$.

对于零质量情形, $H =p$, 上面的小量展开条件不再成立, 


\section{Unitary}
\subsection{算符方法}在$t=0$时, 满足. 所以我们只计算$\frac{\mathrm{d}}{\mathrm{d} t} \left( U^{\dagger} U \right) $. 

由Schr\"odinger方程, 我们有
\begin{equation}
  \frac{\mathrm{d}}{\mathrm{d} t} U(t,0) = -\frac{\mathrm{i} }{\hbar}\hat{H}(t)U(t,0),
  \quad
  \frac{\mathrm{d}}{\mathrm{d} t}\left( U(t,0) \right) ^{\dagger} = \frac{\mathrm{i} }{\hbar}\left( U(t.0) \right) ^{\dagger}\hat{H}(t)
\end{equation}
所以,
\begin{equation}
  \frac{\mathrm{d}}{\mathrm{d} t}\left( U^{\dagger} U \right)  = \frac{\mathrm{d}}{\mathrm{d} t} (U^{\dagger} )U + U^{\dagger} \frac{\mathrm{d}}{\mathrm{d} t}U
  = \frac{\mathrm{i} }{\hbar} U^{\dagger}\hat{H} U - \frac{\mathrm{i} }{\hbar} U^{\dagger} \hat{H} U =0
\end{equation}
这保证了幺正性.

\subsection{路径积分方法}
路径积分当中我们得到的是一个矩阵元, 
\begin{equation}
  \bra{x_N}U(t_N,t_0)\ket{x_0}
\end{equation}
它的幺正性可以写成
\begin{equation}
  \int \mathrm{d} x_b \left( \bra{x_b}U(t_N, t_0) \ket{x_c}\right) ^{*} \bra{x_b}U(t_N,t_0)\ket{x_a} = \bra{x_c}\ket{x_a} = \delta(x_c - x_a)
\end{equation}
幺正性是对于每个时刻成立的, $t$不是这个算符(矩阵)的角标, 不需要对$t$积分.
\begin{equation}
  \begin{gathered}
    \left( \bra{x_b} U(t_N, t_0) \ket{x_c}\right)^{*} \bra{x_b}U(t_N,t_0)\ket{x_a}
    \\ \\
    =  
    \left( \prod_{n=1}^{N-1} \int \frac{\mathrm{d} p'_{n+ \frac{1}{2}} \, \mathrm{d} x'_n}{2\pi \hbar} \right) \int  \frac{\mathrm{d} p'_{\frac{1}{2}}}{2\pi\hbar} 
    \quad
    \left( \prod_{n=1}^{N-1} \int \frac{\mathrm{d} p_{n+ \frac{1}{2}} \, \mathrm{d} x_n}{2\pi \hbar} \right) \int  \frac{\mathrm{d} p_{\frac{1}{2}}}{2\pi\hbar} 
    \\
    \times 
    \exp\left( 
      - \frac{\mathrm{i} }{\hbar} \sum_{n=0}^{N-1} \left[ p'_{n+ \frac{1}{2}}(x'_{n+1}- x'_n) - \delta t \frac{H(p'_{n+\frac{1}{2}},x'_{n+1}, t_{n+ \frac{1}{2}}') + H(\cdots, x'_{n},\cdots)}{2} \right] 
      \right)
    \\
    \times 
    \exp\left( 
      \frac{\mathrm{i} }{\hbar} \sum_{n=0}^{N-1} \left[ p_{n+ \frac{1}{2}}(x_{n+1}- x_n) - \delta t \frac{H(p_{n+\frac{1}{2}},x_{n+1}, t_{n+ \frac{1}{2}}) + H(\cdots, x_{n},\cdots)}{2} \right] 
      \right)
  \end{gathered}
\end{equation}
其中$x_a = x_0 ,\ x_b = x_N,\ x_b = x'_0,\ x_c = x'_N,\ t'_{n} = t_{N-n}$

我们考虑其中包含$x_b$的项,
\begin{equation}
  \exp\left[ \frac{\mathrm{i} }{\hbar}\left( 
    -p_{\frac{1}{2}}' x_1' + p_{\frac{1}{2}}' x_b + p_{N - \frac{1}{2}}x_b - p_{N-\frac{1}{2}}x_{N -1}
    + \delta t \frac{H(p_{\frac{1}{2}}', x_1',t_{\frac{1}{2}})- H(p_{N-\frac{1}{2}},x_{N-1},t_{N- \frac{1}{2}})}{2}
   \right)  \right] 
\end{equation}
这个式子当中$H(p_{\frac{1}{2}}', x_1',t_{\frac{1}{2}})- H(p_{N-\frac{1}{2}},x_{N-1},t_{N- \frac{1}{2}})$, 只有与$p$有关的项, 消去了$x_b$. 

于是可以对$x_b$积分得到一个关于$p$的$\delta$函数, 得到的式子为
\begin{equation}
  2\pi \delta\left( \frac{p_{\frac{1}{2}}' + p_{N -\frac{1}{2}}}{\hbar} \right) \exp \left[ \frac{\mathrm{i} }{\hbar} \left( - p_{\frac{1}{2}}' x_1' - p_{N - \frac{1}{2}}x_{N-1} + \delta t \frac{V(p_{\frac{1}{2}}')-V(p_{N - \frac{1}{2}})}{2} \right)  \right] 
\end{equation}

之后可以对$p_{\frac{1}{2}}'$积分, 恰好消去了$\frac{1}{2\pi\hbar}$的系数, 右侧的动能项也被消去. 
\begin{gather}
  \int \frac{\mathrm{d} p_{\frac{1}{2}}}{2\pi \hbar} \, 2\pi \delta\left( \frac{p_{\frac{1}{2}}' + p_{N -\frac{1}{2}}}{\hbar} \right) \exp \left[ \frac{\mathrm{i} }{\hbar} \left( - p_{\frac{1}{2}}' x_1' - p_{N - \frac{1}{2}}x_{N-1} + \delta t \frac{K(p_{\frac{1}{2}}')-K(p_{N - \frac{1}{2}})}{2} \right)  \right] 
  \\
  = \exp \left[ \frac{\mathrm{i} }{\hbar} \left( - p_{N-\frac{1}{2}} x_1' - p_{N - \frac{1}{2}}x_{N-1}  \right)  \right] 
\end{gather}
再对$p_{N-\frac{1}{2}}$积分, 成为$x$的$\delta$函数
\begin{equation}
  \delta\left( \frac{x_1' - x_{N-1}}{\hbar} \right) .
\end{equation}

再重复这个过程, 对$x$积分消去$\delta$函数, 再对$x$积分, 得到关于$p$的$\delta$函数. 再对$p$积分两次消去$\delta$函数并得到$x$的$\delta$函数...

这个过程是$x-p-x-p$d 循环, 由于这里有$2\times (N-1)$个需要积分的$p$, $2\times (N-2) +1$个需要积分的$x$, 最终的式子会在完成对$p$的积分后出现,
最终得到
\begin{equation}
  \delta \left( \frac{x_c - x_a}{\hbar} \right) .
\end{equation}

\end{document}