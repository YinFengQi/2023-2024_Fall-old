% !TeX root = 线性代数.tex

\begin{example}
    计算$A = \begin{bmatrix}
     3 & 0\\
     4 & 5\\
     0 & 0\\
    \end{bmatrix}$的奇异值分解.
    \\
    \textbf{解 } 
    \begin{equation}
      A^{\mathrm{T}}A = \begin{bmatrix}
       3 & 4 & 0\\
       0 & 5 & 0\\
      \end{bmatrix}
      \begin{bmatrix}
       3 & 0\\
       4 & 5\\
       0 & 0\\
      \end{bmatrix}
      = 
      \begin{bmatrix}
       25 & 20\\
       20 & 25\\
      \end{bmatrix}
    \end{equation}
    求它的特征值可以得到
    \begin{equation}
      \det\left( \lambda I - A \right) = \left( \lambda - 25 \right) ^{2} - 400 = 0,
    \end{equation}
    解得$\lambda_1 = 45, \lambda_2 = 5$, 奇异值分别为$\sigma_1 = 3\sqrt{5}, \lambda_2 = 5$, 对应的特征向量分别为
    \begin{equation}
      v_1 = \frac{1}{\sqrt{2}}\begin{bmatrix}
       1\\
       1\\
      \end{bmatrix},
      \quad
      v_2 = \frac{1}{\sqrt{2}}\begin{bmatrix}
       1\\
       -1\\
      \end{bmatrix}.
    \end{equation}
    下面计算$U$,
    \begin{equation}
      \begin{aligned}
        u_1 = \frac{Av_1}{\sigma_1} = \frac{1}{\sqrt{2}} \begin{bmatrix}
            3 & 0\\
            4 & 5\\
            0 & 0\\
           \end{bmatrix}
           \begin{bmatrix}
            1\\
            1\\
           \end{bmatrix}
           =
           \frac{1}{\sqrt{10}}
           \begin{bmatrix}
            1\\
            3\\
            0\\
           \end{bmatrix}
           \\
           u_2 = \frac{Av_2}{\sigma_2} = \frac{1}{\sqrt{2}} \begin{bmatrix}
            3 & 0\\
            4 & 5\\
            0 & 0\\
           \end{bmatrix}
           \begin{bmatrix}
              1\\
              -1\\
           \end{bmatrix}
             =
             \frac{1}{\sqrt{10}}
           \begin{bmatrix}
              -3\\
              1\\
              0\\
           \end{bmatrix}
      \end{aligned}
    \end{equation}
    于是
    \begin{equation}
      U = \frac{1}{\sqrt{10}}
      \begin{bmatrix}
       1 & -3\\
       3 & 1\\
       0 & 0\\
      \end{bmatrix}
    \end{equation}
    但是我们需要添加一个向量$v_3 = \begin{bmatrix}
     0\\
     0\\
     1\\
    \end{bmatrix}$使$U$的列向量构成正交归一基,
\end{example}

奇异值分解的一个几何理解
\begin{equation}
  \begin{cases}
    A A^{\mathrm{T}} u_i = \sigma_i^{2} v_i, & i=1,2,\cdots,r
    \\
    A A^{\mathrm{T}} u_i = 0, &  i=r+1,\cdots,n
  \end{cases}
\end{equation}
有下列事实
\begin{itemize}
    \item $u_i, \ i=1,2,\cdots,r \in \mathbb{R}^{m} $ 且在$C\left( A \right) $中, 是$C\left( A \right) $的一组正交归一基.
    \item $u_i, \ i=r+1,\cdots,n $ 是在$N\left( A^{\mathrm{T}} \right) $中的一组正交归一基, 是$C\left( A \right) $的正交补的一组正交归一基.
\end{itemize}
同样地,
\begin{itemize}
    \item $v_i, i=1,2,\cdots,r$是$C\left( A^{\mathrm{T}} \right) $的一组正交归一基.
    \item $vi, i=r+1,\cdots ,n$是$C\left( A^{\mathrm{T}} \right) $的正交补的一组基.
\end{itemize}
