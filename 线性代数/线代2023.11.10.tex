% !TeX root = 线性代数.tex

\subsection{特征值的应用}
\subsubsection{求$A^{n}$}
若$A$可以对角化, $\Lambda = P^{-1} A P$, 则
\begin{equation}
  A^{n} = P \begin{bmatrix}
    \lambda_1^{n}&0&\cdots&0\\
    0&\lambda_2^{n}&\cdots&0\\
    \vdots&\vdots&\ddots&\vdots\\
    0&0&\cdots&\lambda_{m}^{n}
  \end{bmatrix}
  P ^{-1}.
\end{equation}

\begin{example}
    求斐波那契数列的第$100$项.($F_1=F_2=0, F_{n} = F_{n-1} + F_{n-2}$)
    \\
    \textbf{解 } 令$u_{k} = \begin{bmatrix}
     F_{k+1}\\
     F_{k}\\
    \end{bmatrix}$, 则有
    \begin{equation}
        u_{k+1} = \begin{bmatrix}
            1&1\\
            1&0\\
        \end{bmatrix} u_{k} \equiv A u_k.
    \end{equation}
    求$A$的特征值, 特征向量, 可以得到
    \begin{equation}
      A^{n} = P \begin{bmatrix}
       \left( \frac{1+\sqrt{5}}{2} \right) ^{n} & 0\\
       0 & \left( \frac{1-\sqrt{5}}{2} \right) ^{n}\\
      \end{bmatrix}
      P ^{-1},
    \end{equation}
    \begin{equation}
      P = \begin{bmatrix}
       \frac{1+\sqrt{5}}{2} & \frac{1-\sqrt{5}}{2}\\
       1 & 1\\
      \end{bmatrix},
      \quad
      P ^{-1} = \frac{\sqrt{5}}{5}\begin{bmatrix}
       1 & \frac{\sqrt{5}-1}{2}\\
       -1 & \frac{\sqrt{5}+1}{2}\\
      \end{bmatrix}
    \end{equation}
    于是
    \begin{equation}
        \begin{aligned}
            A ^{n} = \frac{1}{\sqrt{5}}\begin{bmatrix}
             1 & \frac{\sqrt{5}-1}{2}\\
             -1 & \frac{\sqrt{5}+1}{2}\\
              \end{bmatrix}
              \begin{bmatrix}
               \left( \frac{1+\sqrt{5}}{2} \right) ^{n} & 0\\
               0 & \left( \frac{1-\sqrt{5}}{2} \right) ^{n}\\
              \end{bmatrix}
              \begin{bmatrix}
               \frac{1+\sqrt{5}}{2} & \frac{1-\sqrt{5}}{2}\\
               1 & 1\\
              \end{bmatrix}
              \\ \\
              =
              \frac{\sqrt{5}}{5}
              \begin{bmatrix}
               \left( \frac{1+\sqrt{5}}{2} \right) ^{n+1} - \left( \frac{1-\sqrt{5}}{2} \right) ^{n+1} &\left( \frac{1+\sqrt{5}}{2} \right) ^{n} - \left( \frac{1-\sqrt{5}}{2} \right) ^{n} \\
               \left( \frac{1+\sqrt{5}}{2} \right) ^{n} - \left( \frac{1-\sqrt{5}}{2} \right) ^{n} & \left( \frac{1+\sqrt{5}}{2} \right) ^{n-1} - \left( \frac{1-\sqrt{5}}{2} \right) ^{n-1}\\
              \end{bmatrix}
          \end{aligned}
    \end{equation}
    得到
    \begin{equation}
      F_n = \frac{\sqrt{5}}{5} \left[ \left( \frac{1+\sqrt{5}}{2} \right) ^{n} - \left( \frac{1-\sqrt{5}}{2} \right)^{n}  \right] 
    \end{equation}
\end{example}

\subsubsection{求解矩阵线性微分方程}
方程$\frac{\mathrm{d}u}{\mathrm{d} t} = Au$, 将$A$对角化之后可以直接求解, 设$\Lambda = P ^{-1} A P$, 则
\begin{equation}
  \frac{\mathrm{d}(P^{-1}u)}{\mathrm{d} t} = \Lambda P^{-1} u,
\end{equation}
于是解为
\begin{equation}
    u = P e^{\Lambda t} P ^{-1} u_0 = P \begin{bmatrix}
        e^{\lambda_1 t}&0&\cdots&0\\
        0&\mathrm{e}^{\lambda_2 t}&\cdots&0\\
        \vdots&\vdots&\ddots&\vdots\\
        0&0&\cdots&\mathrm{e}^{\lambda_n t}
    \end{bmatrix} P ^{-1} u_0.
\end{equation}

\subsubsection{求解非矩阵二阶微分方程}
对于方程$u'' + B u' + C u$, 其中$B, C$都是数, $u= u\left( t \right) $.
它可以写为下面的形式
\begin{equation}
    \frac{\mathrm{d}}{\mathrm{d} t} \begin{bmatrix}
     u\\
     u'\\
    \end{bmatrix}
    =
    \begin{bmatrix}
     0 & 1\\
     -C & -B\\
    \end{bmatrix}
    \begin{bmatrix}
     u\\
     u'\\
    \end{bmatrix}.
\end{equation}
之后就可以用对角化的方法求解.

\subsubsection{奇异值分解}
对于任意矩阵都有下列的分解
\begin{equation}
    A = U \Sigma V^{\mathrm{T}} ,
\end{equation}
其中$U$为$m \times m$矩阵, $V$为$n \times n$的正交矩阵, $\Sigma$为$m \times n$矩阵, $\Sigma$的对角线上的元素为奇异值, 且奇异值按照从大到小排列, 
\begin{equation}
   \Sigma = 
   \begin{bmatrix}
    \sigma_1^{2} &  &  &  & \\
     & \sigma_2^{2} &  &  & \\
     &  & \ddots &  & \\
     &  &  & 0 & \\
     &  &  &  & \ddots\\
   \end{bmatrix}
\end{equation}



\color{red}$\large \boxed{\text{待补充}}$
\color{black}