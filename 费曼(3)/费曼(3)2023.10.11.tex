% !TeX root = 费曼(3).tex

The non zero vacuum energy will contribute a force, that is, Casimir effect.

\chapter[自旋-1]{Spin-1}
Stern-Gerlach 实验.
\begin{equation}
  \vec{S}^{2} = \hbar^{2}s(s+1), \ s = 0\text{ or } \frac{1}{2} \text{ or } 1 \text{ or } \cdots
\end{equation}
\begin{equation}
  S_z = m\hbar , \ m = -s, -(s - 1), \ldots , s
\end{equation}

下面我们考虑一个实验, 把一束银原子用磁场分成$z$方向自旋不同的三束, 再合并起来. 我们可以选择用挡板挡住其中两束, 得到一个 projector, 或者什么也不做, 得到一个identity.

我们用$ \ket{iS} $来标记 $S$ 方向自旋为 $i$ 的粒子. 
orthonormal relation:
\begin{equation}
  \bra{i}\ket{j} = \delta_{ij}
\end{equation}

Then, the effect of projector and identity can be written as
\begin{equation}
  I : \sum_{k=-1}^{+1} \ket{k T} \bra{kT} = 1
\end{equation}
completeness relation:
\begin{equation}
  \sum_{j=-1}^{+1} \ket{j} \bra{j} = 1.
\end{equation}
我们最后可以发现 Dirac bracket 是一种带有内积的向量.