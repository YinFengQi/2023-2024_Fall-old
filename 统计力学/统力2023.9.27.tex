% !TeX root = 统力.tex

\section{量子初步}

\subsection{单粒子状态的量子描述}
定态的波函数可以分离出时间项
\begin{equation}
  \psi(\vec{x},t) = \psi(\vec{x}) \mathrm{e}^{-\mathrm{i} E t / \hbar}
\end{equation}
满足定态薛定谔方程
\begin{equation}
  \hat{H}\psi = E \psi.
\end{equation}

\begin{example}
    一维无限深方势阱.
    \begin{equation}
      \varepsilon_n = \frac{h^2 n^2}{8m L^2}.
    \end{equation}
    可以由驻波条件得到.

    估算一下能级差, 对于经典系统, 取$L\sim 10^{-2}\text{m}, m\sim 10^{-27}\text{kg}$, 可以计算得到$\frac{h^2}{8mL^2} \sim 10^{-36}\text{J} \ \ll k_B T \sim 10^{-27}\text{J}$.
    可见能量间距是非常小的.
\end{example}

\begin{example}
    三维容器中的自由粒子.
    \begin{equation}
        \varepsilon_{n_1,n_2,n_3} = \frac{h^2}{8m L^2}(n_1^2 + n_2^2 + n_3^2).
      \end{equation}
\end{example}

\begin{example}
    谐振子.
    \begin{equation}
      \varepsilon_n = \hbar \omega \left( n+ \frac{1}{2} \right) .
    \end{equation}
\end{example}

\begin{example}
    转子.
    \begin{equation}
      H = -\frac{\hbar}{2I}\left[ \frac{1}{\sin\theta} \frac{\partial }{\partial \theta} \left( \sin\theta \frac{\partial }{\partial \theta} \right) + \frac{1}{\sin^2 \theta} \frac{\partial^2 }{\partial \phi^2} \right] 
    \end{equation}
    能级
    \begin{equation}
      \varepsilon_l = \frac{\hbar^2}{2I}l(l+1),
    \end{equation}
    简并度
    \begin{equation}
      \omega_l = 2l +1
    \end{equation}
    磁量子数:
    \begin{equation}
      L_z  = \hbar m
    \end{equation}
\end{example}

\section{量子系统的状态}
\paragraph{微观态:} 按照量子态的占据来区分.
\paragraph{宏观态:} 依据能量来区分.

\subsection{等几率假设}
\textbf{Boltzmann等几率假设: 处于平衡态的孤立系统, 各可能微观状态出现的几率相等.}


\chapter{系综理论}
\subsection{系统微观状态的描述}

\subsubsection{经典}
描述经典系统时, 使用广义坐标$q$, 广义动量$p$. 单粒子的相空间为$\mu$空间, 例子的状态可以用$\mu$空间的点来描述. 粒子自由度为$\gamma$, 有$N$个粒子, 总共$f=\gamma N$个自由度. 系统的相空间为$\Gamma$空间, $\Gamma$空间的维数为$2 N \gamma$.

\subsubsection{量子}
使用力学量完全集$\{L,M, \cdots\}$的量子数$\{l,m,\cdots\}$来描述, 每微观态在$\Gamma$空间占据$h^f$的体积.
