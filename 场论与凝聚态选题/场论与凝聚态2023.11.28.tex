% !TeX root = 场论与凝聚态.tex

\section{Mixed Anomaly between $U(1)$ and $\tilde{U}(1)$}
In XY model, the expression of weight is
\begin{equation}
  \text{weight} = \frac{1}{T} \sum_{l} \left( \cos (\mathrm{d} \theta_{l}) -1\right).
\end{equation}
There exists a $U(1)$ global symmetry,
\begin{equation}
  \theta_{v} \to \theta_{v} + \alpha_{v},\quad \alpha \in U(1), \quad \mathrm{d} \alpha = 0 \mod 2\pi.
\end{equation}

We have known the order parameter of $U(1)$ global symmetry is $\mathrm{e}^{\mathrm{i}\theta_{v}}$, which \textbf{transforms with the global $U(1)$ symmetry transformation}.
We can construct invariant observables with order parameters, to determent the correlation scale or other properties of the system. Non-trivial observables should be $\left< \mathrm{e}^{\mathrm{i} \theta} \mathrm{e}^{- \mathrm{i}\theta_{v'}} \right>$ or more generally, 
\begin{equation}
  \left< \mathrm{e}^{ \mathrm{i} \sum_{v} B_v \theta _{v}} \right> =
  \begin{cases}
    0, & B_v \neq \nabla W_l \\
    $non zero$, & B_v = \nabla W \ ({}= \mathrm{d} ^{*} W)
  \end{cases}
\end{equation}
for the later case, we can obtain the following equation by lattice integration by part. If $B_v = \mathrm{d} ^{*} \left( W_l \right)$,
\begin{equation}
  \left< \mathrm{e}^{\mathrm{i}\sum_{v} B_v \theta_{v}} \right> = \left< \mathrm{e}^{\mathrm{i}\sum_{v} \left( \mathrm{d}^{*} \theta_{l} \right)W_l} \right>
\end{equation}
which is an invariant under $U(1)$.

\subsection{XY Model in $U(1)$ Background}
Coupling to a $U(1)$ background field, the weight of XY model becomes
\begin{equation}
  \text{weight} = \frac{1}{T} \sum_{l}  \left[ \cos \left( \mathrm{d} \theta_{l} - q A_l \right) -1 \right].
\end{equation}
In this case, $U(1)$ global symmetry manifests as $U(1)$ gauge invariance, as we require $A_l$ transforms with $\theta_{v}$
\begin{equation}
  \begin{cases}
    \theta_{v} \to \theta_{v} + \alpha _{v}, & \alpha \in  U(1) \\
    A_l \to A_l + \mathrm{d} \alpha _{l}
  \end{cases}
\end{equation}
where now $\alpha _{v}$ is \emph{arbitrary}, without restriction of $\mathrm{d} \alpha _{l} = 0 \mod 2 \pi $.

Note that the transformation $A_{l} \to A_{l} + \mathrm{d} \alpha _{l}$ gives
\begin{equation}
  \mathcal{Z}[A] = \mathcal{Z}[A + \mathrm{d} \alpha],
\end{equation}
which means different backgrounds (with the difference of a total derivation) are equivalent.

This $A_l$ seems like a electromagnetic field, let us check its current behaviour. Take a Fourier transformation (or Poisson resummation), in current representation we get
\begin{equation}
  \prod{l} I_{j l} \left( \frac{1}{T} \right) \mathrm{e}^{\mathrm{i} j_{l} \left( \mathrm{d} \theta_{l} + q A_l \right)}
\end{equation}
we hope that $A_l$ should be $2 \pi $ periodic, thus $q$ must be a integer.\footnote{We can also couple the $\theta _{v}$ to a series of $A_l$'s, in this case, $\frac{q}{q'}$ must be a rational number.}

The operator correlation transforms as 
\begin{equation}
  \left< \mathrm{e}^{\mathrm{i} \theta_{v}} \mathrm{e}^{- \mathrm{i} \theta _{v'}} \right>_{A} = \mathrm{e}^{\mathrm{i} \left( \alpha _{v} - \alpha _{v'} \right) } \left< \mathrm{e}^{\mathrm{i} \theta _{v}} \mathrm{e}^{-\mathrm{i} \theta _{v'}} \right>_{A + \mathrm{d} \alpha}
\end{equation}

\subsection{Villain Model in $U(1)$ Background}
In Villain model, we have made the substitution,
\begin{equation}
  \cos (\mathrm{d} \theta) -1 \to \frac{1}{2} \mathrm{d} \theta^{2}
\end{equation}
thus it has no natural $2 \pi $ periodicity. We add this manually, by introducing a dynamical degree of freedom $m$ to be summed in path integral. The weight of Villain model is
\begin{equation}
  - \frac{1}{2T} \sum_{l} \bigl( \underbrace{\mathrm{d} \theta + 2 \pi m}_{\gamma _{l}} - A \bigr)_{l}^{2}.
\end{equation}
$m$ transforms with $\theta $ and $A_l$,
\begin{equation}
  \begin{cases}
    A_l \to A_l + \mathrm{d} \alpha _{l} + 2 \pi k_{l}, \\
    \theta_{v} \to \theta _{v} + \alpha _{l} + 2 \pi n_v, \\
    m_l \to m_l + k_l - \mathrm{d} n_l,
  \end{cases}
  \quad \alpha _{v} \in (-\pi ,\pi ],\ k_l, n_v \in \mathbb{Z}
\end{equation}
In this case, the vortex previous is no longer well-defined, as
\begin{equation}
  v_p \equiv \frac{\mathrm{d} \gamma _{p}}{2 \pi} = \mathrm{d} m_p \to \mathrm{d} m_p + \mathrm{d} k_p,
\end{equation}
$v_p$ isn't a invariant now.

We need to find a well-defined vortex. A simple choice would be
\begin{equation}
  \bar{v}_{p} \equiv \frac{\mathrm{d} \left( \gamma - A \right)}{2 \pi },
\end{equation}
which is an invariant under gauge transformation,  but it is \emph{no longer an integer}.

Recalling that we have defined a vortex operator $V_p$ under the limit of $u \to+ \infty $, which is coupled with $\theta_{v}$ as 
\begin{equation}
  - \frac{u}{2} \sum_{p} \left( v_p - V_p \right)^{2} \xlongrightarrow[u \to +\infty]{\text{H-S trans.}} \mathrm{i} \sum_{p} \tilde{\theta}_{p} \left( v_p - V_p \right).
\end{equation}
This can be viewed as a Lagrangian constrain multiplier to restrict the vortex number equal to $V_p$.\footnote{Or get the same result by completing the integration getting a $\delta$ function.}
If we let $V_p \to V_p + \mathrm{d} k_p$, the theory is compatible with the transformation. However, $A_l$ and $V_p$ cannot be viewed as independent, i.e., they together form $U(1)$ background. The $U(1)$ field is described by
\begin{equation}
  \left( \mathrm{d} A - 2\pi V \right)_{p} \in \mathbb{R}.
\end{equation}
instead of the previous one $\mathrm{d} A_l \in U\left(1\right)$. This $\mathrm{d} A - 2\pi V$ can be regarded as a Villainized version of electromagnetic field, which indicates that \emph{a $2\pi $ flux cannot be distinguished with a vortex}.

Both two choice are OK for finite $u$, but interesting things happen when $u \to \infty $
We make a Hubbard-Stratonovich transformation and get the presentation of $\tilde{\theta}$. 
\begin{enumerate}
  \item $\displaystyle \mathrm{i} \sum_{p} \tilde{\theta} _{p} \left( \mathrm{d} m - \frac{\mathrm{d} A}{2\pi } \right)_{p}$, $\tilde{\theta}_{p}$ losses $2\pi $ periodicity. There's no global $\tilde{U}(1)$ symmetry and $\tilde{\theta}_{p}$ lives in $\mathbb{R}$ instead.
  \item $\displaystyle \mathrm{i}\sum_{p} \tilde{\theta}_{p} \left( \mathrm{d} m - V \right)_{p}$, $V_p$ is part of $U(1)$ background, and $\tilde{\theta}$ is actually $2\pi $ periodic.
    
    But let us recall how we identified this $\tilde{U}(1)$ symmetry. By lattice integration by part, we convert $\mathrm{i}\sum_{l} \tilde{\theta}_{l} \mathrm{d} m_l$ to $\mathrm{i}\sum_{l} \tilde{\mathrm{d} } \tilde{\theta}_{l*} m_l $. We want to be able to consider all possible $U(1)$ background, but some $V_p$ will explicitly break $\tilde{U}(1)$ when there exists no $L_p \in \mathbb{Z}$ such $V_p = \mathrm{d} L_p$, leading to the result that the $\delta$ function in $\mathcal{Z}$ will never be met, thus
    \begin{equation}
      \mathcal{Z} = 0 \quad \text{for certain backgrounds $V_p$}.
    \end{equation}
\end{enumerate}
This aberrant phenomenon is called \emph{mixed anomaly}.

In conclusion, when coupling with \emph{a non-trivial $U(1)$ background}, a system with $U(1)$ and $\tilde{U}(1)$ symmetry suffers from $\tilde{U}(1)$ explicitly broken.

In traditional QFT, people believe that anomaly is due to non-invariance of path integral measurement, which is, however, presentation dependent and doesn't reveal the true picture.
