\documentclass{standalone}
\usepackage{tikz}
\usepackage{tikz-3dplot}
\usetikzlibrary{math}
\usepackage{ifthen}
\usepackage[active,tightpage]{preview}
\PreviewEnvironment{tikzpicture}
\setlength\PreviewBorder{1pt}
% 
% 借鉴了tikz.net的一个作品, 以下是原作信息:
% 
% 
% 
% File name: differential-of-surface-area.tex
% Description: 
% A geometric representation of the differential of surface area is shown.
% 
% Date of creation: October, 23rd, 2021.
% Date of last modification: October, 9th, 2022.
% Author: Efraín Soto Apolinar.
% https://www.aprendematematicas.org.mx/author/efrain-soto-apolinar/instructing-courses/
% Source: page 120 of the 
% Glosario Ilustrado de Matem\'aticas Escolares.
% https://tinyurl.com/5udm2ufy
%
% Terms of use:
% According to TikZ.net
% https://creativecommons.org/licenses/by-nc-sa/4.0/
% Your commitment to the terms of use is greatly appreciated.
%
\begin{document}
%
\tdplotsetmaincoords{80}{100}
%
\begin{tikzpicture}[tdplot_main_coords,scale=1.5]
	% the function $z = f(x,y)$
	\tikzmath{function funcion(\x,\y)
    {return .25*sin((0.5*\x + 1.5*\y + .6)r)-.2*\x-.3*\y;
    };}
	\pgfmathsetmacro{\step}{pi/10.0} % step size
	\pgfmathsetmacro{\xi}{0}	% initial value for x
	\pgfmathsetmacro{\xf}{1.0*pi}	% final value for x
	\pgfmathsetmacro{\yi}{0}	% initial value for y
	\pgfmathsetmacro{\yf}{1.0*pi}	% final value for y

    \pgfmathsetmacro{\a}{2*\step}     % 画框的范围, 以下
    \pgfmathsetmacro{\b}{3*\step}
    \pgfmathsetmacro{\c}{5*\step}
    \pgfmathsetmacro{\d}{6*\step}


    % 大框
    \filldraw[draw=black,thick,fill=gray!20] 
        plot[domain=\yi:\yf,smooth,variable=\t] 
        ({\xi},{\t},{funcion(\xi,\t)}) 
        --
        plot[domain=\xi:\xf,smooth,variable=\t] 
        ({\t},{\yf},{funcion(\t,\yf)}) 
        --
        plot[domain=\yf:\yi,smooth,variable=\t] 
        ({\xf},{\t},{funcion(\xf,\t)}) 
        --
        plot[domain=\xf:\xi,smooth,variable=\t] 
        ({\t},{\yi},{funcion(\t,\yi)}) 
        -- cycle;
    
	% 曲面上的网格
	\foreach \x in {0,\step,...,\xf}{
		\draw plot[domain=0:\yf,smooth,variable=\t] ({\x},{\t},{funcion(\x,\t)});
	}
	\foreach \y in {0,\step,...,\yf}{
		\draw plot[domain=0:\yf,smooth,variable=\t] ({\t},{\y},{funcion(\t,\y)});
	}

    % 小框
    \filldraw[draw=blue,thick,fill=blue!20] 
        plot[domain=\c:\d,smooth,variable=\t] 
        ({\a},{\t},{funcion(\a,\t)}) 
        --
        plot[domain=\a:\b,smooth,variable=\t] 
        ({\t},{\d},{funcion(\t,\d)}) 
        --
        plot[domain=\d:\c,smooth,variable=\t] 
        ({\b},{\t},{funcion(\b,\t)}) 
        --
        plot[domain=\b:\a,smooth,variable=\t] 
        ({\t},{\c},{funcion(\t,\c)}) 
        -- cycle;

    % 箭头和文字
        \draw[thick,draw=blue,->] (0,2,0)  -- ({2.5*\step}, {5.5*\step},{funcion({2.5*\step}, {5.5*\step})});

        \node[blue] at (0,2.15,.15) {$\Sigma'_{i}$};
        
        \node at (0,1.2,-2.5) {$\displaystyle \Sigma' = \bigcup_{i} \Sigma_{i}'$};
\end{tikzpicture}
\end{document}