% !TeX root = 高等微积分.tex

\section{数列极限}

之前的阿基米德时代的问题(例\ref{阿基米德})中, 我们需要考虑$n$越来越大的时候, $x_n$是否趋近于某个值$L$. 我们需要定义越来越接近这个概念.

\begin{definition}
    所谓一个无穷序列, 是指一个映射$x\colon \mathbb{Z}_{+} \to \mathbb{R}, n \mapsto x_n$, 记为
    \begin{equation}
      \left\{ x_n \right\} _{n=1}^{\infty} = \left\{ x_n \right\} _{n \in \mathbb{Z}_{+}}
    \end{equation}
    称$x_n$为其第$n$项.
\end{definition}

\begin{definition}
    称数列$\left\{ x_n \right\} _{n=1}^{\infty}$以$L$为极限(limit), (记为$\displaystyle \lim_{n \to \infty}x_n = L$)如果对于任何$\varepsilon> 0$, 都存在$n \in \mathbb{Z}_{+}$使得$\forall n>N$总有$|x_n - L| <\varepsilon$.

    也称当$n\to \infty$时, $x_n$趋于$L$.

    这种定义称为$\varepsilon - N$语言.
\end{definition}
``$\left\{ x_n \right\} $以$L$为极限''可以表示为
\begin{equation}
  \forall \varepsilon > 0 ,\exists N \in \mathbb{Z}_{+}\text{使得} \forall n\ge N \text{有} |x_n - L| < \varepsilon.
\end{equation}

``$\left\{ x_n \right\} $不以$L$为极限''可以表示为
\begin{equation}
  \exists \varepsilon>0  \forall N\in \mathbb{Z}_{+} \exists n \le N \text{使}|x_n-L|\ge \varepsilon.
\end{equation}

\begin{definition}
    称$\left\{ x_n \right\} _{n=1}^{\infty}$是收敛的, 如果$\exists $实数$L$, 使$\left\{ x_n \right\} $以$L$为极限.
    否则, 称$\left\{ x_n \right\} $发散.
\end{definition}

``$\left\{ x_n \right\} $收敛''可以表示为
\begin{equation}
  \exists  L \in \mathbb{R} \forall \varepsilon >0 \exists N \in \mathbb{Z}_{+} \forall n \ge  N, \text{有} |x_n-L|<\varepsilon.
\end{equation}

``$\left\{ x_n \right\} $发散''可以表示为
\begin{equation}
  \forall  L \in \mathbb{R} \exists \varepsilon > 0 \forall N\in \mathbb{Z}_{+} \exists n\ge N \text{使}|x_n - L| \ge \varepsilon.
\end{equation}

\begin{example}
    $\displaystyle \lim_{n \to \infty}\frac{1}{n} = 0.$
\end{example}
\begin{proof}
    $\forall \varepsilon > 0$, 取正整数$N > \frac{1}{\varepsilon}$, 则$\forall n\ge  N$有
    \begin{equation}
      |x_n - 0| = \frac{1}{n} \le \frac{1}{N} < \varepsilon.
    \end{equation}
\end{proof}

\begin{example}
    设$a>1$, 求$\displaystyle \lim_{n \to \infty}a^{\frac{1}{n}}$.

    \paragraph{解}
    求证$\displaystyle  \lim_{n \to \infty}a^{\frac{1}{n}} = 1$.
    为此, $\varepsilon>0$, 取$\displaystyle N = \left\lfloor \frac{a-1}{\varepsilon} \right\rfloor + 1$, 则对$\forall n\ge N$都有
    \begin{equation}
      \left( 1+\varepsilon  \right) ^{n} \ge 1+n\varepsilon \ge 1+ N\varepsilon >a.
    \end{equation}
    从而
    \begin{equation}
      1+\varepsilon > \sqrt[n]{a}.
    \end{equation}
    可以得到
    \begin{equation}
      \left| \sqrt[n]{a} -1 \right| < \varepsilon,
    \end{equation}
    验证了
    \begin{equation}
      \lim_{n \to \infty} \sqrt[n]{a} = 1.
    \end{equation}
\end{example}

\paragraph{总结}$\forall a>0$有$\displaystyle \lim_{n \to \infty}a^{\frac{1}{n}} = 1$.

\begin{example}
    $\displaystyle \lim_{n \to \infty} \sqrt[n]{n} = 1$.
\end{example}
\begin{proof}
    $\forall \varepsilon > 0 $取$N$使$\displaystyle \frac{N-1}{2}\varepsilon^2 > 1$, 则对于$\forall n\ge N$有
    \begin{gather}
      (1+\varepsilon)^n = 1 + C_{n}^{1} \varepsilon + C_{n}^{2} \varepsilon^2 + \cdots \ge C_{n}^{2}\varepsilon^{2}.
      \\
      \ge \frac{(n+1)n}{2} \varepsilon^2 
      \\
      \ge \frac{N+1}{2}\varepsilon^2n  >1\cdot n
    \end{gather}
    从而$\sqrt[n]{n} < 1+\varepsilon$, 得到
    \begin{equation}
      |\sqrt[n]{n}-1| < \varepsilon.
    \end{equation}
\end{proof}

\subsection{极限的性质}
\begin{proposition}[充分大指标的项保持极限不等式]\label{充分大指标的项保持极限不等式}
    设$\displaystyle \lim_{n \to \infty}a_n < \lim_{n \to \infty}b_n$, 则$\exists N \in \mathbb{Z}_{+}$使$\forall  n\ge N $有$a_n < b_n$.
\end{proposition}
\begin{proof}
    设$\displaystyle \lim_{n \to \infty}a_n = A < B = \lim_{n \to \infty}b_n$, 取$\varepsilon = \frac{B-A}{2}>0$.

    由$\displaystyle \lim_{n \to \infty}a_n = A$定义知
    \begin{equation}
      \exists N_1 \in \mathbb{Z}_{+} \forall  n\ge N_1 \text{有} |a_n - A| <\varepsilon.
    \end{equation}

    由$\displaystyle \lim_{n \to \infty}b_n = B$定义知
    \begin{equation}
      \exists N_2 \in \mathbb{Z}_{+} \forall  n\ge N_2 \text{有} |b_n - B| <\varepsilon.
    \end{equation}

    取$N = \max\left\{ N_1,N_2 \right\} $, 则$\forall n \ge N$有
    \begin{equation}
      a_n<A+\varepsilon = B -\varepsilon < b_n.
    \end{equation}
\end{proof}

\paragraph{推论}设$\left\{ a_n \right\} $是正数列, 满足$\displaystyle \lim_{n \to \infty} \frac{a_{n+1}}{a_n} = q < 1$, 则$\displaystyle  \lim_{n \to \infty}a_n = 0$.

\begin{proof}
    取$q<r<1$, 则
    \begin{equation}
      \lim_{n \to \infty} \frac{a_{n+1}}{a_n} < \lim_{n \to \infty} r .
    \end{equation}
    由命题\ref{充分大指标的项保持极限不等式}可知, $\exists N \in \mathbb{Z}_{+}$使$\forall n \ge N$有$\displaystyle \frac{a_{n+1}}{a_n} < r$.
    
    从而, $\forall  n> N$, 有
    \begin{equation}
      \frac{a_n}{a_N} = \frac{a_n}{a_{n-1}} \frac{a_{n-1}}{a_{n-2}} \cdots \frac{a_{N+1}}{a_N} < r^{n-N}. 
    \end{equation}
    即有
    \begin{equation}
      a_n<a_N r^{n-N}, \forall n>N.
    \end{equation}
    由于$\frac{1}{r}>1$, 记$\frac{1}{r} = 1+c ,(c>0)$. 这样, 取$N_0>N+ \frac{a_N}{c\varepsilon}$, 对于$\forall n \ge N_0$, 有
    \begin{gather}
        \left( \frac{1}{r} \right) ^{n-N} = (1+c)^{n-N} \ge (n-N) c 
        \\
        \ge (N_0-N)c 
        > \frac{a_N}{\varepsilon}.
    \end{gather}
    可得
    \begin{equation}
      a_n < a_N r^{n-N} < a_N  \frac{\varepsilon}{a_N} = \varepsilon.
    \end{equation}
\end{proof}

上面最后部分是在算等比级数的极限.
\begin{equation}
  \lim_{n \to \infty}r^n = \begin{cases} 
    0, & |r|<1 
    \\ 
    \text{不存在}, & |r| > 1 \text{或} r=-1
    \\
    1 , & r=1
  \end{cases}
\end{equation}

\paragraph{推论}数列极限是唯一的.
\begin{proof}
    反证法. 设$\left\{ a_n \right\} $既以$A$为极限, 又以$B$为极限, 且$a<B$, 从而
    \begin{equation}
      \lim_{n \to \infty}a_n = A<B = \lim_{n \to \infty}a_n.
    \end{equation}
    由命题\ref{充分大指标的项保持极限不等式}可知, 
    \begin{equation}
      \exists N \in Z_{+}, \forall n\ge N \text{满足} a_n > a_n,
    \end{equation}
    矛盾!
\end{proof}

\paragraph{推论}收敛的数列一定有界.
\begin{definition}
    称数列有上界, 若$\exists M \text{使} \forall n, a_n \le  M$.
    称数列有下界, 若$\exists K \text{使} \forall n, a_n \ge  K$.
\end{definition}
\begin{proof}
    设$\displaystyle \lim_{n \to \infty} x_n = L < L +1 = \lim_{n \to \infty} L +1$, 由命题\ref{充分大指标的项保持极限不等式}可知, 
    \begin{equation}
      \exists  N \in  \mathbb{Z}_{+} , \forall n \ge N \text{有} x_n < L+1.
    \end{equation}
    所以
    \begin{equation}
      x_n \le \max\left\{ x_1, \cdots  ,x_N, L+1 \right\}. 
    \end{equation}
    故有上界, 下界同理.
\end{proof}

\paragraph{推论(极限不等式)} 设$a_n\le b_n , \ \forall  n \ge  N_0$, 若$\lim_{n \to \infty}a_n, \ \lim_{n \to \infty} b_n$存在, 则$\displaystyle \lim_{n \to \infty}a_n \le \lim_{n \to \infty}b_n$.
\begin{proof}
    反证法. 设$\displaystyle \lim_{n \to \infty}a_n > \lim_{n \to \infty} b_n$, 由命题\ref{充分大指标的项保持极限不等式}可知, $\exists n \ge N$有$a_n > b_n$, 矛盾!
\end{proof}

\paragraph{注意!} $\le $可过渡给极限式, 但$<$不一定能.

\begin{example}
    $a_n = 0 < b_n = \frac{1}{n}$, 但 $\displaystyle \lim_{n \to \infty} a_n = 0 =\lim_{n \to \infty} b_n$.
\end{example}

\subsection{极限的计算方法}
\subsubsection{从定义直接计算}
\begin{example}
    多项式增长远小于指数增长,
    \begin{equation}
      \lim_{n \to \infty} \frac{n^{k}}{q^{n}} = 0, \quad \text{当}q>1\text{时}.
    \end{equation}
\end{example}
\paragraph{证法一}
\begin{proof}
  记$x_n = \frac{n^{k}}{q^{n}}$, 注意到
  \begin{equation}
    \begin{gathered}
      \lim_{n \to \infty} \frac{x_{n+1}}{x_n} = \lim_{n \to \infty} \frac{\left( n+1 \right) ^{k} q^n}{q^{n+1} n^{k}}
      \\
      = \lim_{n \to \infty} \left( \underbrace{\frac{n+1}{n} \frac{n+1}{n} \cdots \frac{n+1}{n} }_{k \text{个}} \cdot \frac{1}{q}\right) 
      \\
      =\frac{1}{q} < 1
    \end{gathered}
  \end{equation}
  由命题\ref{充分大指标的项保持极限不等式}知$\lim x_n = 0$.
\end{proof}
\paragraph{证法二(从定义验证)}
\begin{proof}
  对$\forall \varepsilon>0$, 取$N \ge \max\left\{ 2k, \frac{(k+1)! 2^k}{a^{k-1} \varepsilon} \right\} $. $\forall n\ge N$有(记$q=1+a, \ a>0$)
  \begin{equation}
    \begin{gathered}
      \frac{n^{k}}{q^{n}} = \frac{n^{k}}{(1+a)^{n}} \le  \frac{n^{k}}{C^{k+1}_{n} a^{k+1}}
      \\
      = \frac{n^{k} (k+1)!}{n(n-1)\cdots (n-k) a^{k+1}}
      \\
      = \frac{(k+1)!}{a^{k+1}} \frac{1}{n} \frac{n}{n-1} \cdots \frac{n}{n-k}
      \\
      < \frac{(k+1)!}{a^{k+1}} \frac{1}{n} 2\cdot 2 \cdots 2
      \\
      = \frac{(k+1)!}{a^{k+1}} 2^k \frac{1}{n} \le \frac{(k+1)!}{a^{k+1}} 2^k \frac{1}{N} < \varepsilon.
    \end{gathered}
  \end{equation}
\end{proof}


\subsubsection{极限的四则运算}
\begin{theorem}
    设$\lim a_n = A, \ \lim b_n = B$, 则
    \begin{gather}
      \lim (a_n + b_n ) = A + B
      \\
      \lim (a_n - b_n ) = A - B
      \\
      \lim a_n  b_n  = A  B
      \\
      \lim  \frac{a_n}{b_n} = \frac{A}{B}\quad \text{(分母不为零)}
    \end{gather}
\end{theorem}
证明中用到三角不等式(绝对值不等式).
\begin{equation}
  |x+y| \le |x| + |y|
\end{equation}
\begin{proof}
    我们只证极限的乘积和商的性质.
    \paragraph{乘积}
    对于任何$\varepsilon > 0$, 
    \begin{equation}\label{乘积中三角不等式}
      |a_n b_n- AB| = |(a_n -A)b_n + A(b_n - B)| \le  |a_n-A|\cdot|b_n| + |A|\cdot |b_n -B|
    \end{equation}
    \begin{itemize}
      \item 由$\left\{ b_n \right\} $收敛知其有界, 即$\exists M$使$|b_n| \le M, \forall n$.
      
      \item 由$\displaystyle \lim_{n \to \infty}a_n =A$知$\exists N_1 \in \mathbb{Z}_{1}, \forall n\ge N_1$有$|a_n -A| < \frac{\varepsilon}{2M}$.
      
      \item 由$\displaystyle \lim_{n \to \infty}b_n =B$知$\exists N_2 \in \mathbb{Z}_{1}, \forall n\ge N_2$有$|b_n -B| < \frac{\varepsilon}{2(|A|+1)}$.
    \end{itemize}

    从而, 令$N =\max\left\{ N_1,N_2 \right\} $, 对$n\ge N$, 代回\eqref{乘积中三角不等式}得
    \begin{equation}
      |a_n b_n -AB| \le  \frac{\varepsilon}{2M} M + |A| \frac{\varepsilon}{2(|A| + 1)} < \frac{\varepsilon}{2} + \frac{\varepsilon}{2}.
    \end{equation}
    这证明了$\lim a_n b_n = AB$.

    \paragraph{商}
    \begin{gather}\label{商中用到三角不等式}
      \left| \frac{a_n}{b_n} - \frac{A}{B} \right|  = \left| \frac{a_n B - b_n A}{b_n B} \right|  = \left| \frac{(a_n-A)B + A(B -b_n)}{b_nB} \right|
      \\
      \le \frac{|a_n-A|}{|b_n|} + \frac{|A| \cdot |B - b_n|}{|b_n| |B|}.
    \end{gather}

    \begin{itemize}
      \item 由$B\neq 0$, 不妨设$B>0$. 由命题\ref{充分大指标的项保持极限不等式}知$\exists M\in \mathbb{Z}_{+}$使$\forall n \ge M$有$b_n > \frac{B}{2}$
      
      \item 由$\lim_{n \to \infty}a_n = A$知$\exists N_2, \forall  n\ge  N_2$有$|a_n -A| < \varepsilon' = \frac{\varepsilon}{2} \frac{B}{2} $
      
      \item 由$\lim_{n \to \infty}b_n = B$知$\exists N_3, \forall n\ge N_3$有$|b_n-B| < \varepsilon'' = \frac{\varepsilon}{2} \frac{\frac{1}{2}B^2 }{|A|+1} $
    \end{itemize}
  
    $\forall \varepsilon > 0$取$N = \max\left\{ N_1, N_2,N_3 \right\} $, 对$\forall n\ge  N$有(代回\eqref{商中用到三角不等式})
    \begin{equation}
      \left| \frac{a_n}{b_n} - \frac{A}{B} \right|  \le  \frac{\frac{\varepsilon}{2} \frac{B}{2}}{\frac{B}{2}} + \frac{|A| \frac{\varepsilon}{2} \frac{\frac{1}{2} B^2}{|A|+1}}{\frac{B}{2} B} < \frac{\varepsilon}{2} + \frac{\varepsilon}{2} = \varepsilon.
    \end{equation}
\end{proof}