% !TeX root = 高等微积分.tex

\color{red}{$\large \boxed{\text{待补充}}$}
\color{black}

\begin{definition}
    称$I$是区间, 如果$I$是$\mathbb{R}^{1}$的凸集. 即$\forall P, A \in I$则线段$PQ \subseteq I$.
\end{definition}

\begin{proposition}
    考虑$\inf I = m$(约定, 若无下界令$m$为符号$-\infty$), $\sup I = M$(同, 无上界: $+\infty$).

    $\forall m < x < M$, 有$x \in I$. 由$\frac{m+x}{2} > m = \inf I$, 知$\exists x_1 \in I$使 $\frac{m+x}{2} > x_1$. 同理$\exists x_2 \in I$使 $\frac{x+M}{2} < x_2$. 于是$x_1 < x < x_2$, 由$I$的凸性知$\left[ x_1, x_2 \right]  \in I$.
    特别的, $x \in I$.

    这样
    \begin{equation}
        \left( m.M  \right) \subseteq  I \subseteq \left[ m,M \right] \implies I = \left( m, M \right) \cup \left( \text{端点集的集合} \right) 
    \end{equation} 
\end{proposition}


\begin{theorem}[反函数定理]
    设$I$是区间, 设$f \colon I \to \mathbb{R}$是连续单射, 则
    \begin{itemize}
        \item $f\left( I \right) $是区间.
        \item $f^{-1} \colon f\left( I \right) \to I$是连续的.
    \end{itemize}
\end{theorem}

\begin{proof}[证明第一条]
    $\forall f\left( x_1 \right) , f\left( x_2 \right) \in f\left( I \right) $, 对$f\left( x_1 \right) , f\left( x_2 \right) $, 使用介值定理, 线段$f\left( x_1 \right) f\left( x_2 \right) $包含在$f\left( I \right) $中 $\implies f\left( I \right)$是$\mathbb{R}$的区间.
\end{proof}

\begin{proof}[证明第二条]
    由$f$连续单射可知$f$严格单调, 不妨设$f$严格递增
    \begin{enumerate}
        \item 当$y_0$是$f\left( I \right) $的内点时, 设$f\left( x_0 \right) = y_0$, 则$x_0$是$I$的内点, 则$\forall \varepsilon> 0 $取$0 < \varepsilon^{1} < \varepsilon$使$x \pm \varepsilon^{1} \in  I$.
        
        令$\delta = \min \{ y_0 - f\left( x_0 - \varepsilon^{1} \right) , f\left( x_0 + \varepsilon^{1} \right) - y_0 \}$, 这使得$y$介于$f\left( x_0 + \varepsilon^{1} \right) , f\left( x_0 - \varepsilon^{1} \right) $之间. 
        
        由介值定理可知, $\exists x \in \left( x_0 - \varepsilon^{1}, x_0 + \varepsilon ^{1} \right) $使得 $f\left( x \right)  = y$. 于是$\left| x - x_0 \right| < \varepsilon^{1} \implies \left| f\left( x \right) - f\left( x_0 \right)  \right| < \varepsilon$, 即$f$在$x_0$处连续.
        
        \item 当$y_0$是$f\left( I \right) $的端点, 不妨设是左端点. 类似令$f\left( x_0 \right) = y_0 $, 则$x_0$也是$I$的左端点. 令$\delta = f\left( x_0 + \varepsilon^{1} \right) - y_0$, 则$\forall y \in \left( y_0, y_0 + \delta \right) $, 由介值定理可知, $\exists x \in \left( x_0, x_0 + \varepsilon^{1} \right) $使得$f\left( x \right) = y$. 于是$\left| x - x_0 \right| < \varepsilon^{1} \implies \left| f\left( x \right) - f\left( x_0 \right)  \right| < \varepsilon$, 即$f$在$x_0$处连续.
    \end{enumerate}
\end{proof}

\begin{example}[幂函数]
    分情况三种:
    \begin{itemize}
        \item $f\left( x \right) = x^{n}, \ n \in \mathbb{Z}_{+}, \ [0,+\infty) \to  \mathbb{R}$显然$f$连续且严格递增\footnote{这可以用二项式展开
        \begin{equation}
          \left( x + h \right) ^{n} = \sum C_{n}^{l} x^{n - l} h^{l} > x^{n}
        \end{equation}}.

        由反函数定理可知, $f$有连续的反函数$f^{-1}\colon [0, +\infty) \to [0, +\infty)$, 记为$f^{-1} \left( y \right) = y^{\frac{1}{n}}\colon \mathbb{R}_{\ge 0} \to \mathbb{R}_{\ge 0}$

        \item 定义$x^{\frac{m}{n}}\colon \mathbb{R}_{\ge 0}\to \mathbb{R}_{\ge 0}$为$x^{\frac{m}{n}} = \left( x^{\frac{1}{n}} \right) ^{m}$.
        
        \item $\forall \alpha \in  \mathbb{R} / \mathbb{Q}$, 取有理数序列$\{ \alpha_{n} \} \to \alpha$, 定义$x^{\alpha} = \lim_{n \to \infty} x^{\alpha_{n}}$.
    \end{itemize}
\end{example}

\begin{example}[对数函数]
    $f\left( x \right) = \mathrm{e}^{x}$有连续反函数$f^{-1}\left( y \right) \xlongequal{\text{记为}} \ln y\colon \mathbb{R}_{+} \to  \mathbb{R}_{+}$.
\end{example}

\begin{example}[反三角函数]
    $f\left( x \right) = \sin x \colon \left[ -\frac{\pi}{2}, \frac{\pi}{2} \right] \to  \mathbb{R}$严格递增且连续, $f$有连续的反函数$f^{-1}\left( y \right) \xlongequal{\text{记为}} \arcsin y\colon \left[ -1, 1 \right] \to  \left[ -\frac{\pi}{2}, \frac{\pi}{2} \right] $.

    类似地, $\arccos x \colon [-1, 1 ] \to [0, \pi],\  \arctan x \colon (-\infty, +\infty) \to \left( - \frac{\pi}{2}, \frac{\pi}{2} \right) $.
\end{example}


\subsection[无穷小量与无穷大量]{无穷小量与无穷大量\protect\footnote{这描述的是某些函数具有特定的极限行为, 并不是某一个数是无穷小/无穷大}}

\begin{definition}
    称$x \to x_0$时, $f\left( x \right) $是无穷小量$\iff \lim_{x \to x_0} f\left( x \right) = 0 $, 正/负无穷大量同理.
\end{definition}

\begin{example}
    $x \to 0$时, $x, \sin x, x^{n}\ (n\ge 1)$是无穷小量, $\ln \left| x \right| , \frac{1}{x}$是无穷大量.
\end{example}

引入无穷小量/无穷大量的比较, 
\begin{itemize}
    \item $f$是比$g$更高阶的无穷小量$\iff \lim_{x \to 0} \frac{f\left( x \right) }{g\left( x \right) } = 0$.
    \item $f$是与$g$同阶的无穷小量$\iff \lim_{x \to 0} \frac{f\left( x \right) }{g\left( x \right) } \in \mathbb{R} / \{ 0 \}$.
    \item $f$是与$g$等价的无穷小量$\iff \lim_{x \to 0} \frac{f\left( x \right) }{g\left( x \right) } = 1$.
\end{itemize}
无穷大量的比较同理.

在计算极限时, 可以把某乘积因子替换为与之等价的无穷大/无穷小量, 不改变极限值. 因为
\begin{equation}
    \lim \left( f\left( x \right) h\left( x \right)  \right) = \lim \left( \frac{f(x)}{g(x)} g(x)h(x) \right)   
\end{equation}

\section{微分与导数}

\begin{definition}
    $f$在某点处的导数定义为
    \begin{equation}
      \lim_{h \to 0} \frac{f\left( x_0 + h \right) - f\left( x \right) }{h} \xlongequal{\text{极限换元}} \lim_{x \to x_0} \frac{f\left( x \right) - f\left( x_0 \right) }{x - x_0}.
    \end{equation}
    记为$f '\left( x \right) $
\end{definition}