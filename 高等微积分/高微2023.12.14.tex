% !TeX root = 高等微积分.tex

由此可以得到, 对于两个剖分 $P_1, P_2$, 它们的并为 $P = P_1 \cup P_2$,$P$ 是 $P_1, P_2$ 的公共加细, 于是有
\begin{equation}
  L(P_2,f) \le L(P, f) \le U(P,f) \le U(p_1,f)
\end{equation}
所以任何一个下和小于任何一个上和.
令 
\begin{equation}
  U = \inf \{ U(p,f) | \forall \text{剖分$P$}\}, \quad L = \sup \{ F(P,f) | \forall \text{剖分$P$} \}
\end{equation}

\begin{proposition}
  $L = U \iff $ $ \forall \varepsilon > 0$ , $\exists \text{剖分}P$使得
  \begin{equation}
    U(P,f) - L(P,f) < \varepsilon.
  \end{equation}
\end{proposition}
\begin{proof}
  从充分和必要性两方面来证明.

  ``$\implies $'':
  设 $U = L$, 则由确界定义, 
  \begin{equation}
    \begin{aligned}
      \exists U(P_1, f) & < U + \frac{\varepsilon}{2}, \\
      \exists L(P_2,f) < L - \frac{\varepsilon}{2},
    \end{aligned}
  \end{equation}
  令 $P = P_1 \cup P_2$, 则
  \begin{equation}
    U(P,f) - L(P,f) \le U(P_1,f) - L(P_2,f) < \left( U + \frac{\varepsilon}{2} \right) - \left( L - \frac{\varepsilon}{2} \right) < \varepsilon.
  \end{equation}

  ``$\impliedby $'':
  设右边成立, 则
  \begin{equation}
    U-L \le  U(P,f) - L(P,f) < \varepsilon
  \end{equation}
  由 $\varepsilon$ 任意性知 $U = L$.
\end{proof}

\begin{theorem}
  $f$ 可积, 当且仅当 $L = U$, 并且有
  \begin{equation}
    \int_{a}^{b} f(x) \, \mathrm{d}x = L = U.
  \end{equation}
\end{theorem}

\begin{theorem}
  连续函数一定可积.
\end{theorem}
\begin{proof}
  设 $f \in C\left( [a,b] \right)$, 则 $f$ 在 $[a,b]$ 上一致连续.
  从而 $\varepsilon > 0, \ \exists \delta > 0$, 使得 $\forall \left| x-y \right| < \delta$ 有
  \begin{equation}
    \left| f(x) - f(y) \right| < \frac{\varepsilon}{b-a}
  \end{equation}
  这样, 取剖分 $P$ 的每个 $I_i$, $\forall x, y \in I_i$ 有 $\left| x-y \right|< \varepsilon$,
  \begin{equation}
    \implies \left| f(x) - f(y) \right| < \frac{\varepsilon}{b-a}
  \end{equation}
  于是
  \begin{equation}
    U(P,f) - L(P,f) = \sum_{i}  \left( \sup_{I_i} f(x) - \inf_{I_i} f(x) \right) \left| I_i \right| \le \frac{\varepsilon}{b-a} \left( b-a \right) = \varepsilon.
  \end{equation}
\end{proof}

\begin{theorem}
  单调有界函数皆可积.
\end{theorem}
\begin{proof}
  不妨设 $f$ 递增, 取剖分 $P$ 为均匀剖分, 
  \begin{equation}
    x_i = a + \frac{i}{N} \left( b-a \right), \quad (0 \le i \le N)
  \end{equation}
  从而
  \begin{equation}
    \begin{aligned}
      U(P,f) - L(P,f) & = \sum_{i=1}^{N} \left( \sup _{I_i} f(x) - \inf _{I_i} f(x) \right)\left| I_i \right|
    \\
                      & = \sum_{i=1}^{N} \left( f(x_{i}) - f(x_{i-1}) \right) \frac{b-a}{N} \\
                      & = \left( f(b) - f(a) \right) \frac{b-a}{N} < \varepsilon
    \end{aligned}
  \end{equation}
  上式的最后一个不等号只要 $N$ 足够大就可以做到.
\end{proof}

\begin{theorem}[Riemann-Lebesgue]
  $f$ 在 $[a,b]$ 上可积, 当且仅当 $f$ 在 $[a,b]$ 上有界, 且 $f$ 在 $[a,b]$ 上的所有间断点构成的集合是零测的.
\end{theorem}

\begin{definition}
  称 $\mathbb{R}$ 的子集 $E$ 是零测集, 如果 $\forall \varepsilon > 0$, 存在可数多个开区间 $I_1, I_2, \cdots$ 使得 
  \begin{equation}
    \bigcup_{i=1}^{+ \infty} I_i \supseteq E
  \end{equation}
  且
  \begin{equation}
    \sum_{i=1}^{+ \infty} \left| I_i \right| < \varepsilon. 
  \end{equation}
\end{definition}

\begin{example}
  \begin{itemize}
    \noindent
    \item 有限集皆是零测. 
    \item 可数集也是零测的.
  \end{itemize}
\end{example}

上面第二条来自于以下命题:
\begin{proposition}
  可数多个零测集之并也是零测集.
\end{proposition}
\begin{proof}
  设 $E_1, E_2, \cdots $ 都是零测集, 由 $E_n$ 零测的定义可知, 存在一族开区间 $I_{n 1}, I_{n 2}, \cdots $, 使得
  \begin{equation}
    \bigcup_{i=1}^{+\infty } I_{n i} \supseteq E, \text{ 且 } \sum_{i=1}^{\infty }  \left| I_{ni} \right| < \frac{\varepsilon}{2^{n}}
  \end{equation}
  这样, 
\end{proof}
