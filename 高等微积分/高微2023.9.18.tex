% !TeX root = 高等微积分.tex

\section{微积分简介}

\subsection{阿基米德时代}
问题:设$D= \left\{ (x,y)\middle| 
 a\le x\le b, \quad 0\le y\le h(x)
 \right\} $求曲边梯形$D$的面积$\operatorname{area}\left( D \right) $.

特例: $a=0$, 剖分$D = \bigcup D_i$, 分点$x_i = \frac{ib}{n}$

\begin{itemize}
    \item 算$\operatorname{area}\left( D_i \right)\simeq \left( x_i - x_{i-1} \right) h(\xi_i), \quad \xi_i \in \left[ x_{i-1}, x_i \right]  $
    \item 求和
    \begin{equation}
      \operatorname{area}\left( D \right) \simeq \sum_{i=1}^{n} \left( x_{i-1}-x_i \right) h(\xi)
    \end{equation}
    \item 相信随着剖分越来越细, 上述近似越来越好
\end{itemize}




\begin{example}
    $h(x)=x^{2}$
    \begin{align}
      \operatorname{area}\left( D \right) 
      & \simeq \sum_{i=1}^{n} \frac{b}{n} h\left( \xi_i = x_i \right) =\sum_{i=1}^{n} \frac{b}{n}\left( \frac{ib}{n} \right) ^{2} = \frac{b^{3}}{n^{3}}\sum_{i=1}^{n} i^{2}
      \\
      & = \frac{b^{3}}{n^{3}} \frac{1}{6} n(n+1)(2n+1)
      \\
      & = \frac{b^{3}}{6} \left( 1+ \frac{1}{n} \right) \left( 2+ \frac{1}{n} \right) 
      \\
      & = \frac{b^{3}}{6} \left( 2 + \frac{3}{n} + \frac{1}{n^2} \right) 
      \xlongequal{\text{记为}}_{} x_n
    \end{align}
    研究: 当$n$越大时, $x_n$最终会靠近哪个常值$L$ 
\end{example}

\begin{example}\label{阿基米德}
    $h(x)=x^{k}, \quad \left( k\ge 2 \right) $
    相应的
    \begin{equation}
      \operatorname{area}\left( D \right) \simeq \frac{b^{k+1}}{n^{k+1}}\sum_{i=1}^{n} i^k
    \end{equation}
    更接近哪个数$L$? 对于更一般$h$, 以上计算更加复杂.
\end{example}

\subsection{Newton时代}
上述问题反问题:
已知面积函数$S(a)$, 如何求高度?

$x$流动到$x+o$, 
\begin{align}
  S(x+o)-S(x) \simeq o\cdot h(x)
  \\
  \implies h(x) \simeq \frac{S(x+o)-S(x)}{o} \quad \text{(流数法)}
\end{align}
相信当$o$越接近零, 此近似越好.

\begin{example}
    $S(a) = a^{m}, \quad \left( m \in \mathbb{Z}_+ \right) $
    \begin{equation}
      \implies h(x) \simeq \frac{(x+o)^m-x^m}{o}
    \end{equation}
    使用牛顿二项式公式
    \begin{equation}
      \left( x+y \right) ^{m} = x^{m} + C_{m}^{1} x^{m-1}y + \cdots  + C_{m}^{m} y^{m}
    \end{equation}
    带入, 得到
    \begin{equation}
      h(x)\simeq C_{m}^{1} x^{m-1} +C_{m}^{2} x^{m-2} o  \cdots  + C_{m}^{m} o^{m-1}
      \xlongequal{\text{令$o$等于零}} mx^{m-1}
    \end{equation}
\end{example}

由此可知, 例\ref{阿基米德} 答案为$S(a) = \frac{1}{k+1} a^{k+1}$

\begin{itemize}
    \item 从高度函数得到面积称作积分$S(b) = \int_{0}^{b} h(x) \, \mathrm{d}x$
    \item 从面积函数得到高度函数称作求导 $h(x) = S'(x)$
\end{itemize}


进行一个循环, 可以得到
\begin{equation}
  \left( \int_{0}^{x} h(\xi) \, \mathrm{d}\xi \right) ' = h(x)
\end{equation}
和
\begin{equation}
  \int_{0}^{b} S'(x) \, \mathrm{d}x = S(b) - S(0)
\end{equation}



\section{集合与映射}
\begin{definition}
    设$X, Y$是集合, 所谓$X$到$Y$的一个映射是指如下的数据
    
    对于$X$中的每一个元素$x$, 指定$Y$中唯一的元素(记为$f(x)$)与之对应.
    记此映射为
    \begin{equation}
      f \colon X \rightarrow Y
    \end{equation}
    (这个符号直到1940年代才开始出现, 标志着范畴论的开始)

    称$X$为$f$的定义域 domain, $Y$为$f$的陪域 co-domain.
    \begin{equation}
      \forall A \subseteq X, \quad f(A) = \left\{ y \in Y \middle|\exists a \in A \text{使} y= f(a) \right\} 
    \end{equation}
    称之为$A$在$f$下的像集.特别的, 称$f(X) = \operatorname{Im}(f)$为$f$的值域或像集.
\end{definition}

%定义原像集
\begin{definition}
    原像集. 对$V \subseteq Y$, 定义在$f$下的原像集
    \begin{equation}
        f^{-1}(V) = \left\{ x \in X \middle| f(x) \in V \right\}
        = \bigcup_{y \in V} F_{y}
    \end{equation}
    对于$V$的补集$V^c$显然有,
    \begin{equation}
      f^{-1}(V^c) = \left( f^{-1}(V) \right) ^c
    \end{equation}

\end{definition}
显然有
\begin{equation}
  f^{-1} \left( A \cup B \right) = f^{-1}(A) \cup f^{-1}(B)
\end{equation}
%%%%% 缺

\subsection{映射的性质}

\begin{itemize}
    \item 映射可复合.设$f\colon X\rightarrow Y,\ g \colon Y \rightarrow Z$, 可定义复合映射$g \circ f \colon X \rightarrow Z$,
    \begin{equation}
      g \circ f(x) = g\left( f(x) \right) , \quad \forall x\in X
    \end{equation}

    \item 映射的复合满足结合律.设$f\colon X\rightarrow Y,\ g\colon Y\rightarrow Z, \ h\colon Z\rightarrow W $, 则
    \begin{equation}
      h \circ \left( g\circ f \right) = (h \circ g) \circ f
    \end{equation}
    证明是直接的.

    \item 对于集合$X$有一个恒同映射, $\operatorname{id}_X \colon X \to X$, 定义为 $\operatorname{Id}_X(x) = x, \quad \forall x \in X$
    
    \item 恒同映射是映射复合的单位, 即$\forall f\colon X\rightarrow Y $有
    \begin{equation}
      \operatorname{id}_Y \circ f = f = f \circ \operatorname{id}_X
    \end{equation}
\end{itemize}


对于两个集合$X,Y$, 存在一个集合
\begin{equation}
  \operatorname{Hom}(X,Y) = \left\{ \text{从$X$到$Y$的映射} \right\} 
\end{equation}



